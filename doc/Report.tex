\documentclass[10pt,a4paper]{article}

\usepackage[a-1b]{pdfx} %high quality pdf
\usepackage[style=iso]{datetime2}
\usepackage{numprint}
\usepackage{palatino}
\usepackage{authblk}
\usepackage[margin=0.5in]{geometry}
\usepackage{hyperref}
\usepackage{graphicx}
\usepackage{listings}
\usepackage{mathtools}
\usepackage{amsmath}
\usepackage{amsfonts}
\usepackage{amssymb}
\usepackage{amsthm}
\usepackage{multirow}

\pdfgentounicode=1
%paragraph indentation
\setlength{\parindent}{2em}
\renewcommand{\baselinestretch}{1.5}
\nplpadding{2}

\title{%
        {\Huge Programming Project Report}\\
        {\Large COT5405-Analysis of Algorithms}
}
\author{Naman Arora}
\affil{UFID: 3979-0439}
\date{\today}

\begin{document}
\maketitle
%\begin{figure}[h!]
%	\includegraphics[width=\textwidth]{pics/xview.jpg}
%        \caption{\textit{X's} view}
%        \label{xview}
%\end{figure}
\section*{Preamble}
The given problem, hereinafter referred to as \textit{posed problem}, is to find a square/rectangle in a given 2D matrix of integers with the largest area that has all the integers greater than or equal to a given parameter \textit{`h'}.
The problem bears a stark resemblance to another very famous problem of finding the largest square/matrix of \textit{1's} in terms of its area given a 2D binary matrix, hereinafter referred to as \textit{binary matrix problem}.

Given the algorithms to solve the \textit{binary matrix problem}, the \textit{posed problem} can be transformed into the former using the transformation such that,
\begin{equation} \label{trans}
        a`_{i, j} = \begin{cases}
                1 \quad & a_{i, j} \geq h,\\
                0 \quad & otherwise
        \end{cases}
\end{equation}
where,\\
$ a`_{i, j} $ is the $ (i, j)^{th} $ element of the transformed matrix and\\
$ a_{i, j} $ is the $ (i, j)^{th} $ element of original matrix.

The program starts with the \textit{`main'} function.
This function parses the commandline arguments which must include the name of the \textit{testcase} file along with an optional log file.

The \textit{testcase} file is a space separated file that contains $ |rows| + 1 $ lines.
The first line in \textit{testcase} file must have the $ |rows| $, $ |cols| $ and $ h $ in this order.
The next $ |rows| $ lines each have $ |cols| $ space separated integers that represent the input integer 2D array.

The optional \textit{logfile} is expected to be a path to a non-existent file which would be written to by the \textit{timer} module.
If the file at the path of \textit{logfile} argument already exists, it \textit{WILL} be overwritten.
If the argument for \textit{logfile} is empty, \textit{STDOUT} is used as the \textit{logfile} instead.

The \textit{`execute'} function is responsible for reading the file line by line and creating the 2D array in the memory of the program.
The 2D array is a vector of vectors of integers.
This function is also responsible for transforming the grid according to \ref{trans} explained above.

\section*{Algorithm 1}
The first algorithm is responsible for finding the largest square, in terms of area, of \textit{1s} in the given 2D binary matrix.
The recurrent relation of the approach is of the form,
\begin{equation} \label{alg1}
        d_{i, j} = \begin{cases}
                0 \quad & a_{i, j} = 0,\\
                1 + min\{d_{i-1, j}, d_{i, j-1}, d_{i-1, j-1}\} \quad & i,\ j \ne 0,\\
                1 \quad & otherwise
        \end{cases}
\end{equation}
where,\\
$ d_{i, j} $ represents the width of a square of \textit{1s} with its lower right corner at position $ (i, j) $.
The algorithm aims to select the width which comes out to be largest among all the $ d_{i, j} $ where $ 0 \leq i < |rows| $ and $ 0 \leq j < |cols| $.
The algorithm exploits the fact that in order to increase the dimension of a square by a single unit, all of the elements of the current row, column and diagonal being examined have to be \textit{1}.

The reason to why the algorithm is based on recursively solving subproblems has to do with the fact that the problem has \textit{\textbf{optimal substructure}}.
At each step, $ d_{i, j} $ merely represents whether the width of a square with its bottom right corner at $ (i, j) $ comprises of a single element, that is itself, or a larger square of \textit{1s} which was determined in a preceding step.
The proof of optimality can be obtained by \textit{proof by induction}.
It is trivial to see that in case $ a_{i-1, j-1} $, $ a_{i-1, j} $ or $ a_{i, j-1} $ had been \textit{0}, $ d_{i-1, j-1} $, $ d_{i-1, j} $ or $ d_{i, j-1} $ would also have become 0.
In this case, the value of $ d_{i, j} $ would solely depend on itself as a subproblem, or in other words $ d_{i, j} $ would depend on $ a_{i, j} $.
Conversely, if all of $ a_{i-1, j-1} $, $ a_{i-1, j} $ or $ a_{i, j-1} $ had been \textit{1} then, $ d_{i-1, j-1} $, $ d_{i-1, j} $ or $ d_{i, j-1} $ would have represented that and $ d_{i, j} $ would have extended on the previously formed square.

The \textbf{\textit{time complexity}} of the algorithm is $ \mathcal{O}(|rows| \times |cols|) $.
This follows from the analysis that, for each $ a_{i, j} $, the algorithm takes $ \mathcal{O}(1) $ time to calculate $ d_{i, j} $.
The 2D matrix can be traversed element by element in $ \mathcal{O}(|rows| \times |cols|) $ time.
Also, the $ d^{max}_{i, j} $ can be kept track of during the calculation of each $ d_{i, j} $ in $ \mathcal{O}(1) $.
Each time $ d_{i1, j1} > d^{max}_{i, j} $, an update to the co-coordinates can be done in $ \mathcal{O}(1) $.
Thus, upper bound for the algorithm is $ \mathcal{O}(|rows| \times |cols|) \times \mathcal{O}(1) \times \mathcal{O}(1) = \mathcal{O}(|rows| \times |cols|) $.

\subsection*{Task 1}
Task 1 implements the algorithm described in \ref{alg1} in a \textit{top down}, dynamic programming approach.
The algorithm begins with input of the 2D binary array and creates a $ |rows| \times |cols| $ sized table, \textit{cache}, to keep track of $ d_{i,j} $.
For each element $ (i, j) $, the algorithm recurses to calculate $ d_{i+1, j} $, $ d_{i, j+1} $ and $ d_{i+1, j+1} $ and store them in \textit{cache}.
If in case $ i = |rows| $ or $ j = |cols| $ during any call of the recursion, that turn immediately returns \textit{0} signifying out of bounds.
The existence of \textit{cache} basically avoids the recalculation of each $ d_{i, j} $ and can be fetched in $ \mathcal{O}(1) $ time.
\textit{Cache} table here is \textbf{\textit{memoization table}}.
The \textbf{\textit{space complexity}} of the algorithm is of the order $ \mathcal{O}(|rows| \times |cols|) $ which is the creation of \textit{cache}.

\subsection*{Task 2}
Task 2 implements the algorithm explained in \ref{alg1} in a \textit{bottom up} approach.
This is a more true to nature implementation of this algorithm.
At each step, the algorithm calculates $ d_{i, j} $ for an element $ a_{i, j} $ by directly adhering to \ref{alg1}.
The only implementation detail is that it just maintains a table \textit{dp} of size $ \times (|cols|+1) $.
Each element $ (i, j) $ just needs a directly predeceasing elements $ d_{i-1, j} $, $ d_{i, j-1} $ and $ d_{i-1, j-1} $ and nothing beyond that.
The algorithm oscillates between the second and third rows of \textit{dp} depending on $ i\%2 $ and utilizes the other row as its predecessor.
Thus, the \textbf{\textit{space complexity}} of this implementation is $ \mathcal{O}(2 \times |cols|) = \mathcal{O}(|cols|) $.

\section*{Algorithm 2}
The second algorithm is a brute force mechanism to find the largest rectangle in terms of area with all \textit{1s} given a 2D binary array.
The algorithm creates windows of top left and bottom right corners and enumerates all the rectangles within a that window.
Enumerating each window, represented by $ (i_{1}, j_{1}) $ and $ (i_{2}, j_{2}) $ as top left and bottom right corners, takes an order of $ \mathcal{O}(|rows|^{2} \times |cols|^{2}) $ to enumerate.
Within each window, every rectangle is enumerated and its area is calculated.
If the current area, $ s $ happens to be greater than $ s^{max} $, the $ s^{max} $ is updated and the current top left and bottom right corners are recorded.

The \textbf{\textit{time complexity}} of this algorithm is of the order $ \mathcal{O}(|rows| \times |cols|) \times \mathcal{O}(|rows| \times |cols|) \times \mathcal{O}(|rows| \times |cols|) $ which is $ \mathcal{O}(|rows|^{3} \times |cols|^{3}) $.

\subsection*{Task 3}
The task 3 is an implementation of the brute force algorithm.
It inputs the 2D binary matrix and runs three series of \textit{for} loops, each containing two \textit{for} loops.
The first series enumerates $ (i_{1}, j_{1}) $, the top left corner.
Within that, the second series enumerates $ (i_{2}, j_{2}) $ such that $ i_{1} \leq i_{2} < |rows| $ and $ j_{1} \leq j_{2} < |cols| $.
The third series enumerates all the rectangles within the given window, such the for $ (i_{3}, j_{3}) $, $ i_{1} \leq i_{3} \leq i_{2} $ and $ j_{1} \leq j_{3} \leq j_{2} $.
The \textbf{\textit{space complexity}} is $ \mathcal{O}(1) $ since no extra space is used.

\section*{Algorithm 3}
The third and final algorithm needs to find the largest area rectangle of \textit{1s} in a given 2D binary matrix.

\subsection*{Task 5 \textit{(Bonus)}}

\section*{Task 4}

\end{document}
